\documentclass[journal,12pt,twocolumn]{IEEEtran} 

\usepackage{setspace}
\usepackage{gensymb}
\singlespacing
\usepackage[cmex10]{amsmath}

\usepackage{amsthm}
\usepackage{mathrsfs}
\usepackage{txfonts}
\usepackage{stfloats}
\usepackage{bm}
\usepackage{cite}
\usepackage{cases}
\usepackage{subfig}

\usepackage{longtable}
\usepackage{multirow}

\usepackage{enumitem}
\usepackage{mathtools}
\usepackage{steinmetz}
\usepackage{tikz}
\usepackage{circuitikz}
\usepackage{verbatim}
\usepackage{tfrupee}
\usepackage[breaklinks=true]{hyperref}
\usepackage{graphicx}
\usepackage{tkz-euclide}

\usetikzlibrary{calc,math}
\usepackage{listings}
  \usepackage{color}                                              %%
  \usepackage{array}                                              %%
  \usepackage{longtable}                                          %%
  \usepackage{calc}                                               %%
  \usepackage{multirow}                                           %%
  \usepackage{hhline}                                             %%
  \usepackage{ifthen}                                             %%
  \usepackage{lscape}                                             %%

\renewcommand\thesubsubsectiondis{\thesubsectiondis.\arabic{subsubsection}}

  
\usepackage{multicol}
\usepackage{chngcntr}

\DeclareMathOperator*{\Res}{Res}

\renewcommand\thesection{\arabic{section}}
\renewcommand\thesubsection{\thesection.\arabic{subsection}}
\renewcommand\thesubsubsection{\thesubsection.\arabic{subsubsection}}

\renewcommand\thesectiondis{\arabic{section}}
\renewcommand\thesubsectiondis{\thesectiondis.\arabic{subsection}}
\hyphenation{op-tical net-works semi-conduc-tor}
\def\inputGnumericTable{}                           %%

\lstset{
%language=c,
frame=single,
breaklines=true,
columns=fullflexible
}

\begin{document}

\newcommand{\BEQA}{\begin{eqnarray}}
\newcommand{\EEQA}{\end{eqnarray}}
\newcommand{\define}{\stackrel{\triagle}{=}}
\bibliographystyle{IEEEtran}
\raggedbottom
\setlength{\parident}{0pt}
\providecommand{\mbf}{\mathbf}
\providecommand{\Pr}[1]{\ensuremath{\Pr\left(#1\right)}}
\providecommand{\qfunc}[1]{\ensuremath{Q\left(#1\right)}}
\providecommand{\sbrak}[1]{\ensuremath{{}\left[#1\right]}}
\providecommand{\lsbrak}[1]{\ensuremath{{}\left[#1\right.}}
\providecommand{\rsbrak}[1]{\ensuremath{{}\left.#1\right]}}
\providecommand{\brak}[1]{\ensuremath{\left(#1\right)}}
\providecommand{\lbrak}[1]{\ensuremath{\left(#1\right.}}
\providecommand{\rbrak}[1]{\ensuremath{\left.#1\right)}}
\providecommand{\cbrak}[1]{\ensuremath{\left\{#1\right\}}}
\providecommand{\lcbrak}[1]{\ensuremath{\left\{#1\right.}}
\providecommand{\rcbrak}[1]{\ensuremath{\left.#1\right\}}}
\theoremstyle{remark}
\newtheorem{rem}{Remark}
\newcommand{\sgn}{\mathop{\mathrm{sgn}}}
\providecommand{\abs}[1]{\vert#1\vert}
\providecommand{\res}[1]{\Res\displaylimits_{#1}} 
\providecommand{\norm}[1]{\lVert#1\rVert}
%\providecommand{\norm}[1]{\lVert#1\rVert}
\providecommand{\mtx}[1]{\mathbf{#1}}
\providecommand{\mean}[1]{E[ #1 ]}
\providecommand{\fourier}{\overset{\mathcal{F}}{ \rightleftharpoons}}
%\providecommand{\hilbert}{\overset{\mathcal{H}}{ \rightleftharpoons}}
\providecommand{\system}{\overset{\mathcal{H}}{ \longleftrightarrow}}
	%\newcommand{\solution}[2]{\textbf{Solution:}{#1}}
\newcommand{\solution}{\noindent \textbf{Solution: }}
\newcommand{\cosec}{\,\text{cosec}\,}
\providecommand{\dec}[2]{\ensuremath{\overset{#1}{\underset{#2}{\gtrless}}}}
\newcommand{\myvec}[1]{\ensuremath{\begin{pmatrix}#1\end{pmatrix}}}
\newcommand{\mydet}[1]{\ensuremath{\begin{vmatrix}#1\end{vmatrix}}}
\numberwithin{equation}{subsection}
\makeatletter
\@addtoreset{figure}{problem}
\makeatother
\let\StandardTheFigure\thefigure
\let\vec\mathbf
\renewcommand{\thefigure}{\theproblem}
\def\putbox#1#2#3{\makebox[0in][l]{\makebox[#1][l]{}\raisebox{\baselineskip}[0in][0in]{\raisebox{#2}[0in][0in]{#3}}}}
     \def\rightbox#1{\makebox[0in][r]{#1}}
     \def\centbox#1{\makebox[0in]{#1}}
     \def\topbox#1{\raisebox{-\baselineskip}[0in][0in]{#1}}
     \def\midbox#1{\raisebox{-0.5\baselineskip}[0in][0in]{#1}}
\vspace{3cm}
\title{Assignment1(AI1103)}
\author{SABNE KETAN SANTOSH - CS20BTECH11043}
\maketitle
\newpage
\bigskip
\renewcommand{\thefigure}{\theenumi}
\renewcommand{\thetable}{\theenumi}

 \begin{center}
     \section{\textbf{Problem statement}}
 \end{center}
A die is thrown again and again until three sixes are obtained.Find the probability of obtaining the third six in the sixth row of a die

\begin{center}
    \section{\textbf{Solution}}
\end{center}
 Let P be the probability of getting third six in sixth row.
 \newline
 Now,we will solve this question using multiplication rule in probability,For that we will use a table. Here we are in restricted mode because we want 3rd six in sixth trial. So, there must be only two sixes in first five trials. Also, we will have to think about combinations of getting two sixes in first five trials.Let I be probability in individual trial(of getting 6 or not getting 6). So,if we got 6 then I=$\frac{1}{6}$ and if we did not get 6 then I=
 $\frac{5}{6}$.
\begin{table}[ht]
\centering
\begin{tabular}{|c|c|c|c|c|c|c|c} \hline
    Trial No & 1 & 2 & 3 & 4 & 5 & 6 \\ \hline
    I & $\frac{1}{6}$ &$\frac{5}{6}$ &$\frac{5}{6}$&$\frac{5}{6}$&$\frac{1}{6}$&
   $\frac{1}{6}$ \\ \hline
    \end{tabular}
    \caption{ Table of probability }
    \label{table:1}
\end{table}

So, now using multiplication rule, we can say 
\begin{align}
    P={5 \choose 2} \times \brak{\frac{5}{6}}^3 \times \brak{\frac{1}{6}}^3 \\
    P= 10 \times \brak{\frac{5}{6}}^3 \times \brak{\frac{1}{6}}^3 \\
    P= \frac{1250}{6^6}
\end{align}

\begin{equation}
    \boxed{P=0.0267}
\end{equation}
   
So,the probability of getting third six in sixth row is 0.0267.
\end{document}
